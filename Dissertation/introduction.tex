\chapter*{ВВЕДЕНИЕ}							% Заголовок
\addcontentsline{toc}{chapter}{ВВЕДЕНИЕ}	% Добавляем его в оглавление

В данной работе представлен датасет для классификации позы собак, эффективный способ его получения с помощью частичного привлечения и людей и компьютерного зрения, а также способ использовать этот датасет для создания системы мониторинга поведения собаки в течении дня на компактном, обособленном устройстве. Такое устройство поможет понять, есть ли у питомца проблемы с активностью, спит ли собака весь день или остаётся активным даже когда хозяина нет дома. Такое устройство необходимо для владельцев собак, которым необходимо отследить начинающиеся признаки деперессии и нарастающей пассивности у собак.

В ходе решения задачи был создан набор данных для распознавания базовых поз собаки: стоячей, сидячей и лежачей. В датасете содержится 4500 изображений, по 1500 на каждый класс, при этом эти данные тщательным образом очищены и позволяют даже крупным нейронным сетям, которые требуют более внушительных объёмов данных, получать хорошие результаты.

Эта работа является предпосылкой к более глубинному изучению эмоций и сигналов домашних животных, в ходе которых планируется дальше использовать системы компьютерного зрения.