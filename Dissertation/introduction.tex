\chapter*{Введение}							% Заголовок
\addcontentsline{toc}{chapter}{Введение}	% Добавляем его в оглавление

Собаки - давние друзья человека. В США 50\% домов имеют хотя бы одну собаку, а в одной Калифорнии живёт 77 миллионов собак. \cite{dog_ownership} Собаки постоянно коммуницируют с человеком. Но они не умеют говорить, как это делают люди. Зато у них есть другой способ коммуникации - их собственное тело. И если человек передаёт информацию главным образом с помощью голоса, и только дополняет её языком тела, то у собак с точностью до наоборот. Понимать язык собак несложно, там нет сложных языковых конструкций, зато есть простые сигналы. 

Опытные собачники утверждают что постоянно общаются со своими домашними животными. Иметь стойкую коммуникацию крайне важно. Однажды собака может спасти своего хозяина от катастрофы, нередки случаи когда собаки предупреждали хозяев о пожарах, цунами и утечках газа, ведь они способны воспринимать более широкий спектр звуковых частот и имеют намного более острый нюх. В то же время, отсутствие коммуникации с животным может сильно навредить им. Главной причиной возврата собак в приюты всё ещё является мискоммуникация с хозяином, из-за которых животное может повредить имущество или быть агрессивным, непослушным и лаять\cite{reasons_dogs_return_to_shelters}. А иногда животные могут подать сигнал о том, что они больны, и тогда есть риск потерять даже жизнь животного, не распознав его. 

Но решений по распознаванию сигналов собак ещё не существует. В магазине приложений для iPhone имеется всего одно приложение, которое связано с распознаванием чего-либо у собак. Это приложение умеет распознавать породу собаки. Важно заметить, что и у IBM\cite{ibm_dog} и у Microsoft\cite{microsoft_dog} имеются небольшие проекты по тому же самому распознаванию пород собак. Причина по которой так распространено распознавание пород собак, а распространение сигналов собак не распространено не в том, что это менее востребовано. Как раз, наоборот. Дело в том, что для распознавания породы собак, Стенфордский университет в 2011 году создал набор данных, благодаря которому, это возможно решать. Вся индустрия машинного обучения и искусственного интеллекта сильно нуждается в хороших данных для решения проблем. И эта работа нацелена, как раз, на создание таких данных. Конечно, до создания датасета по распознаванию самих сигналов собак потребуется намного больше ресурсов, чем есть у автора, но те данные, которые стали продуктом исследования, сильно приближают тот день, когда такое распознавание станет возможным.

В этой работе был создан набор данных для распознавания простейших поз собаки - поз, когда собака стоит, сидит и лежит. В нём содержится 4500 изображений, по 1500 на класс. Этого достаточно, чтобы обучить некоторые популярные архитектуры нейронных сетей. Также в последней главе описаны приёмы для обучения даже достаточно глубоких архитектур нейронных сетей, которые обычно требуют значительно большего объёма данных.