\chapter*{Введение}							% Заголовок
\addcontentsline{toc}{chapter}{Введение}	% Добавляем его в оглавление

Между собаками и людьми постоянно происходит коммуникация. Несмотря на то, что собаки не умеют говорить, как это делают люди, собаки используют другой способ коммуникации - их собственное тело. Человек передаёт информацию главным образом с помощью голоса и только дополняет её языком тела. А собаки, наоборот. Понимать язык собак несложно, в нём нет сложных языковых конструкций. И люди научились выделять некоторые из собачьих жестов, которые называются сигналами. 

Иметь стойкую коммуникацию с животным крайне важно. Известно множество случаев, когда собаки спасал своего хозяина катастрофы. Чаще всего это пожары, цунами и утечки газа. Это обусловлено тем, что ухо собаки способно воспринимать более широкий спектр звуковых частот. А нос собак гораздо более чувствительный и распознаёт больше различных ароматов.

В то же время, отсутствие коммуникации с животным может сильно навредить животным. Ежегодно 10\% собак, взятых из приютов, возвращаются обратно. Чаще всего это происходит из-за неправильно налаженной коммуникации.\cite{reasons_dogs_return_to_shelters} Последствия её могут включать повышенную агрессию животного, непослушание и порчу имущества. Также животные могут подать сигнал о том, что они больны, и тогда есть риск потерять жизнь животного, если такой сигнал не распознать. 

Компьютерных решений по распознаванию сигналов собак ещё не существует. В магазине приложений для iPhone имеется только одно приложение на похожую тему. Это приложение распознаёт породу собаки. 

Причина, по которой существует продукт по распознаванию собачьих пород - это наличие хороших данных. В 2011 году, Фей Фей ли и её помощники из Стенфордского университета \cite{KhoslaYaoJayadevaprakashFeiFei_FGVC2011} создали набор данных, благодаря которому эту задачу возможно решать. Вся индустрия машинного обучения и искусственного интеллекта сильно нуждается в хороших данных для решения проблем. И эта работа нацелена, как раз, на создание таких данных. 

В этой работе был создан набор данных для распознавания простейших поз собаки - поз, когда собака стоит, сидит и лежит. В нём содержится 4500 изображений, по 1500 на класс. Этого достаточно, чтобы обучить некоторые популярные архитектуры нейронных сетей. Также в работе описаны приёмы для обучения даже достаточно глубоких архитектур нейронных сетей, которые обычно требуют значительно большего объёма данных.
