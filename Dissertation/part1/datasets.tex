Насколько известно автору, по состоянию на Май 2020 года в открытом доступе содержатся только следующие наборы данных с изображениями собак:
\begin{enumerate}
  \item ImageNet \cite{imagenet} - Считается крупнейшим датасетом по классификации всего. Насчитывает более 1 миллиона изображений и больше 1000 классов изображений для классификации, начиная от машин, заканчивая собаками. Часто используется для оценки производительности систем компьютерного зрения а также для предобучения нейронных сетей при недостаточных данных.
  \item Stanford Dog Dataset \cite{KhoslaYaoJayadevaprakashFeiFei_FGVC2011} - Подраздел ImageNet. В нём содержится 20000 изображений собак 120 различных пород. Разметка осуществлялась для дальнейшей классификации собак по породам. Разные породы имеют различное количество изображений.
  \item OpenImageDataset \cite{openimages} - ближайший аналог ImageNet по назначению и реализации, за тем лишь исключением что он создавался с уклоном в детекцию объектов, поэтому все изображения там чуть большего размера, и на каждом изображения может быть несколько объектов, в том числе, разного класса. К каждому объекту прилагается информация о его ограничивающей рамке.
  \item DogCentric Activity Dataset \cite{yumi2014first} - Датасет с видеозаписями различных занятий собаки от лица самой собаки. Целью является классификация активности.
  \item Jena Action Recognition Dataset \cite{jena} - Коллекция видеозаписей с дистанционно управляемым роботом-собакой SONY ERS-7 Aibo. Создавалась для оценивания систем распозанавания. В ней есть видеозаписи, координаты ограничивающих рамок робота на каждом кадре и данные для калибровки.
\end{enumerate}
Все эти датасеты достаточно хорошо размечены. Важно заметить что в изображения собак в ImageNet, Stanford Dog Dataset и OpenImageDataset пересекаются, так что суммарно по этим трём датасетам имеется всего 20000 изображений собак, столько же сколько и в Stanford Dog Dataset.