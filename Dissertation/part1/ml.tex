Машинное обучение предполагает построение математических моделей, помогающих понять данные. "Обучение" вступает в игру, когда мы даем этим моделям настраиваемые параметры, которые могут быть адаптированы к наблюдаемым данным; таким образом, программу можно считать " обученной" на основе данных. После того, как эти модели подогнаны под ранее наблюдаемые данные, их можно использовать для прогнозирования и понимания аспектов вновь наблюдаемых данных.

\subsubsection{Категории машинного обучения}\label{ml}
На самом фундаментальном уровне машинное обучение можно разделить на два основных типа: обучение с учителем и без него.

%здесь график с методами машинного обучения

Обучение с учителем включает в себя некое моделирование взаимосвязи между измеряемыми характеристиками данных и некоторой маркировкой, связанной с данными; после определения этой модели она может быть использована для нанесения меток на новые, неизвестные данные. Далее это подразделяется на задачи классификации и регрессионные задачи: в классификации метки являются дискретными категориями, а в регрессии - непрерывными величинами. 

% side by side регрессия и классификация

Обучение без учителя включает в себя моделирование свойств данных без привязки к какой-либо метке, и часто описывается как "позволить данным говорить самим за себя". Эти модели включают такие задачи, как кластеризация и уменьшение размерности. Алгоритмы кластеризации идентифицируют различные группы данных, в то время как алгоритмы уменьшения размерности ищут более сжатые представления данных.

% кластеризация

Кроме того, существуют так называемые методы с частичным привлечением учителя, которые находятся где-то между обучением под наблюдением и обучением без наблюдения. Методы с частичным привлечением учителя часто бывают полезны, когда метки имеются лишь для части данных.


%\subsubsection{Регрессия}



%\subsubsection{Классификация}




%Вся идея машинного обучения состоит в следующих нескольких идеях. 

%1. Есть среда, которая генерирует данные
%2. Мы, как эксперты, пытаемся описать от каких факторов зависит изменение этих данных
%3. Мы выбираем математческую модель, которая могла бы с помощью наших факторов генерировать результат, идентичный результатам среды при тех же факторах
%4. Специальная программа, оптимизатор, подгоняет коэффициенты у нашей модели на основе собранных данных, таким образом, чтобы ответы сходились

%Предположим, мы хотим предсказать вес человека по его росту. Конечно, мы знаем что помимо роста на вес влияет множество факторов, но нас устроит и приблизительный ответ. Мы собираем информацию о росте и весе нескольких людей, собрав их в одну табличку. 

%\subsection{Методы компьютерного зрения}
%
%\subsection{Свёрточные нейронные сети}
%Свёрточные нейронные сети работают по тому же принципу что и фильтры для изображений. Кроме того, что их много, они организуются в слои из этих фильтров и сами обучаются тому какую форму им принять.
%\subsubsection{Классификация изображений}

%\subsubsection{Локализация объектов}

%\subsubsection{Обнаружение объектов}

%\subsubsection{Pose Estimation}