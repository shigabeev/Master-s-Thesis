Перед тем как говорить о компьютерном зрении, следует уделить внимание тому что такое машинное обучение и какие в нём есть сложности и ограничения.

% Классификация, кластеризация и регрессии
% Классификация изображений
% Object localization
% Object Detection
% Pose estimation
%


\subsection{Методы машинного обучения}
Прежде чем рассмотреть детали различных методов машинного обучения, давайте начнем с рассмотрения того, чем является машинное обучение, а чем нет. Машинное обучение часто относят к подобласти искусственного интеллекта, но я считаю, что такая классификация часто может вводить в заблуждение с первой же попытки. Изучение машинного обучения, безусловно, возникло в результате исследований в этом контексте, но в научном применении методов машинного обучения более полезно думать о машинном обучении как о средстве построения моделей данных.

По сути, машинное обучение предполагает построение математических моделей, помогающих понять данные. "Обучение" вступает в игру, когда мы даем этим моделям настраиваемые параметры, которые могут быть адаптированы к наблюдаемым данным; таким образом, программу можно считать " обученной" на основе данных. После того, как эти модели подогнаны под ранее наблюдаемые данные, их можно использовать для прогнозирования и понимания аспектов вновь наблюдаемых данных.

\subsubsection{Категории машинного обучения}
На самом фундаментальном уровне машинное обучение можно разделить на два основных типа: обучение с учителем и без него.

%здесь график с методами машинного обучения

Обучение с учителем включает в себя некое моделирование взаимосвязи между измеряемыми характеристиками данных и некоторой маркировкой, связанной с данными; после определения этой модели она может быть использована для нанесения меток на новые, неизвестные данные. Далее это подразделяется на задачи классификации и регрессионные задачи: в классификации метки являются дискретными категориями, а в регрессии - непрерывными величинами. 

% side by side регрессия и классификация

Обучение без учителя включает в себя моделирование свойств данных без привязки к какой-либо метке, и часто описывается как "позволить данным говорить самим за себя". Эти модели включают такие задачи, как кластеризация и уменьшение размерности. Алгоритмы кластеризации идентифицируют различные группы данных, в то время как алгоритмы уменьшения размерности ищут более сжатые представления данных.

% кластеризация

Кроме того, существуют так называемые методы с частичным привлечением учителя, которые находятся где-то между обучением под наблюдением и обучением без наблюдения. Методы с частичным привлечением учителя часто бывают полезны, когда имеются лишь некоторые метки.


%\subsubsection{Регрессия}



%\subsubsection{Классификация}


