Обязательным процессом любой задачи, связанной с машинным обучением является получение и разметка данных. Разметкой данных, в зависимости от объёма, можно заниматься как самостоятельно, так и с помощью наёмного труда. 

\subsection{Самостоятельная разметка данных} \label{self_labeling}
Практика показывает, что на самостоятельный просмотр тысячи изображений уходит около часа. В зависимости от наличия необходимых инструментов, на разметку этих изображений может уйти в полтора раза больше времени. Обычно, за один подход человек может просмотреть около 2 тысяч изображений, после чего ему требуется отдых. За один день обычно размечается около 4 тысяч изображений.

К достоинствам самостоятельной разметки можно отнести:
\begin{itemize}[wide]
    \item Надёжность -- возможность полностью контролировать результат;
    \item Независимость -- нет нужды полагаться на других людей или сервисы;
    \item Дешевизна -- не нужно никому платить;
    \item Возможность ознакомиться с набором данных.
\end{itemize}
И действительно, ознакомившись с набором данных, можно увидеть его недостатки, или наоборот, особенности, которые могут помочь решить задачу.

Недостатки:
\begin{itemize}[wide]
    \item Разметка больших объёмов данных изнуряет;
    \item Это самый медленный способ получить данные;
    \item Часто приходится создавать инструменты для разметки самостоятельно.
\end{itemize}


\subsection{Разметка данных с помощью сторонних сервисов} \label{toloka}
Когда самостоятельная разметка невозможна или занимает слишком много времени, можно воспользоваться специальными сервисами для разметки данных. Такими являются Яндекс.Толока и Amazon Mechanical Turk.

Достоинства:
\begin{itemize}[wide]
    \item Скорость - из-за большого количества пользователей у сервиса на разметку даже больших датасетов редко уходит больше часа;
    \item Встроенные в сервисы инструменты для разметки;
    \item Относительная дешевизна.
\end{itemize}
Недостатки:
\begin{itemize}[wide]
    \item Качество разметки крайне низкое;
    \item Для сложных заданий требуется составлять обучение;
    \item Требуется опыт в составлении заданий на платформе;
    \item Невозможность разметки коммерчески секретных данных.
\end{itemize}
Главным недостатком таких сервисов является то, что пользователи не знают ничего о задаче автора. Поэтому задания надо составлять максимально точно, и заставлять пользователей проходить обучение прежде чем решать задачу. Более того, часть пользователей могут вместо правильных ответов давать быстрые, и за этим тоже необходимо следить.

\subsection{Разметка данных с помощью наёмного штата} \label{label_staff}
Обычно, серьёзные команды на рынке машинного обучения для разметки данных используют собственные наёмные команды.

Достоинства:
\begin{itemize}[wide]
    \item Высокое качество разметки;
    \item Возможность получать обратную связь;
    \item Возможность лично и устно формулировать задания штату;
    \item Возможность подписать соглашение о неразглашении.
\end{itemize}
Недостатки:
\begin{itemize}[wide]
    \item Цена. Это самый дорогой способ.
\end{itemize}
Часто такие команды нанимают под однотипные задачи. Например, если есть постоянный поток данных с камер автомобилей, такая команда может размечать данные специально для компании. В таком случае можно обеспечить постоянную нагрузку на штат, а сотрудники будут опытными в решении задачи.



\subsection{Сравнение методов разметки}
В таблице \ref{tab:labeling_comparison} можно сравнить разные способы разметки данных по приведённым характеристикам.


\begin{table}[H]
\centering
\captionsetup{justification=centering}
\caption{\label{tab:labeling_comparison} Сравнительная таблица методов разметки.}
\begin{tabularx}{\textwidth}{| Y | Y | Y | Y |}\hline
     Характеристика & Самостоятельно & Сервисы & Штат\\\hline
     Качество       & \bullet\bullet\bullet         & \bullet\bullet          & \bullet \\ 
     Скорость       & \bullet         & \bullet\bullet          & \bullet\bullet\bullet \\ 
     Оперативность  & \bullet\bullet\bullet         & \bullet\bullet          & \bullet \\ 
     Низкая цена    & \bullet\bullet\bullet         & \bullet          & \bullet\bullet \\ 
     Обратная связь & \bullet\bullet\bullet         & \bullet\bullet          & \bullet \\
     Масштабируемость & \bullet         & \bullet\bullet          & \bullet\bullet\bullet  \\\hline
\end{tabularx}
\end{table}
