Обязательным процессом любой задачи связанной с машинным обучением является получение и разметка данных. Разметкой данных в зависимости от объёма можно заниматься как самостоятельно, так и с помощью наёмного труда. 

\subsection{Самостоятельная разметка данных} \label{subsect1_3_1}
Нет ничего страшного в том чтобы разметить несколько тысяч изображений. Практика показывает, что на самостоятельную разметку тысячи изображений при наличии необходимых инструментов уходит от 20 минут до часа.

К достоинствам самостоятельной разметки можно отнести:
\begin{itemize}
    \item Надёжность - возможность полностью контролировать результат
    \item Независимость - нет нужды полагаться на других людей или сервисы
    \item Дешевизна - не нужно никому платить
    \item Возможность ознакомится с набором данных
\end{itemize}
И действительно, ознакомившись с набором данных, можно увидеть его недостатки, или наоборот, особенности, которые могут помочь решить задачу.

Недостатки:
\begin{itemize}
    \item Разметка больших наборов данных изнуряет
    \item Это самый медленный способ получить данные
    \item Часто приходится создавать инструменты для разметки самостоятельно
\end{itemize}


\subsection{Разметка данных с помощью сторонних сервисов} \label{subsect1_4_2}
Когда самостоятельная разметка невозможна или занимает слишком много времени, можно воспользоваться специальными сервисами для разметки данных. Такими являются Яндекс.Толока и Amazon Mechanical Turk.

Достоинства:
\begin{itemize}
    \item Скорость - из-за большого количества пользователей на разметку даже больших датасетов редко уходит больше часа.
    \item Встроенные в сервисы инструменты для разметки
    \item Относительная дешевизна.
\end{itemize}
Недостатки:
\begin{itemize}
    \item Качество разметки крайне низкое
    \item Для сложных заданий требуется составлять обучение
    \item Требуется опыт в составлении заданий на платформе
    \item Невозможность разметки коммерчески секретных данных
\end{itemize}
Главным недостатком таких сервисов является то, что пользователи не знают ничего о задаче автора. Поэтому задания надо составлять максимально точно, и заставлять пользователей проходить обучение прежде чем решать задачу. Более того, часть пользователей могут вместо правильных ответов давать быстрые, и за этим тоже необходимо следить.

\subsection{Разметка данных с помощью наёмного штата} \label{subsect1_4_3}
Обычно, серьёзные команды на рынке машинного обучения для разметки данных используют собственные наёмные команды.

Достоинства:
\begin{itemize}
    \item Высокое качество разметки
    \item Возможность получать обратную связь
    \item Возможность лично и устно формулировать задания штату
    \item Возможность подписать соглашение о неразглашении
\end{itemize}
Недостатки:
\begin{itemize}
    \item Цена. Это самый дорогой способ
\end{itemize}
Часто такие команды нанимают под однотипные задачи. Например, если есть постоянный поток данных с камер автомобилей, такая команда может размечать данные специально для компании. В таком случае можно обеспечить постоянную нагрузку на штат, а сотрудники будут опытными в решении задачи.