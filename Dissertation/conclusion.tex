\chapter*{Заключение}						% Заголовок
\addcontentsline{toc}{chapter}{Заключение}	% Добавляем его в оглавление

%% Согласно ГОСТ Р 7.0.11-2011:
%% 5.3.3 В заключении диссертации излагают итоги выполненного исследования, рекомендации, перспективы дальнейшей разработки темы.
%% 9.2.3 В заключении автореферата диссертации излагают итоги данного исследования, рекомендации и перспективы дальнейшей разработки темы.
%% Поэтому имеет смысл сделать эту часть общей и загрузить из одного файла в автореферат и в диссертацию:

В результате данной работы была создана система распознавания позы собаки по видеопотоку. Дополнительно к этому была такая решена важная подзадача, как подготовка датасета для классификации базовых поз собак.

Среди количественных результатов следует отметить что поза собаки при соблюдении всех условий классифицируется с точностью 93\%. Скорость работы всей системы на телефоне iPhone 11 составляет 24 кадра в секунду, т.е. готова работать в реальном времени.

Помимо этого, был собран датасет, содержащий 4200 изображений, а также надёжный и быстрый способ расширения этого датасета.

Его можно использовать для распознавания позы собаки, а также, его можно дополнять изображениями других поз и попытаться распознать некоторые сигналы домашних животных.

Что касается распознавания более тяжёлых поз - с этими проблемами возможно бороться. Например, в OpenPose\cite{openpose} успешно борются с окклюзиями конечностей людей и множественными объектами, а значит, при определённой упорности, получится перенести эти технологии в распознавание поз собак.