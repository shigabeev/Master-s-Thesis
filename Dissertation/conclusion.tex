\chapter*{Заключение}						% Заголовок
\addcontentsline{toc}{chapter}{Заключение}	% Добавляем его в оглавление

%% Согласно ГОСТ Р 7.0.11-2011:
%% 5.3.3 В заключении диссертации излагают итоги выполненного исследования, рекомендации, перспективы дальнейшей разработки темы.
%% 9.2.3 В заключении автореферата диссертации излагают итоги данного исследования, рекомендации и перспективы дальнейшей разработки темы.
%% Поэтому имеет смысл сделать эту часть общей и загрузить из одного файла в автореферат и в диссертацию:

В результате данной работы был создан датасет для классификации базовых поз собак. Его можно использовать для распознавания позы собаки, а ещё лучше, его можно дополнить изображениями других поз и попытаться распознать некоторые сигналы домашних животных. При этом полнота распознавания не так важна. Даже если мы пропустим 60\% сигналов от домашних животных и примем с уверенностью только 40\% сигналов - это на 40\% лучше, чем то, что мы имеем сейчас.

Конечно, метод сбора данных в этой статье спорный, но благодаря ему было возможно создание действительно чистых данных, которые нейронная сеть может успешно распознать. Это позволяет двигаться дальше, имея рабочий прототип. 

Что касается распознавания более тяжёлых поз - с этими проблемами возможно бороться. Например, в OpenPose\cite{openpose} успешно борются с окклюзиями конечностей людей и множественными объектами, а значит, при определённой упорности, получится перенести эти технологии в распознавание поз собак.