\chapter*{ЗАКЛЮЧЕНИЕ}						% Заголовок
\addcontentsline{toc}{chapter}{ЗАКЛЮЧЕНИЕ}	% Добавляем его в оглавление

%% Согласно ГОСТ Р 7.0.11-2011:
%% 5.3.3 В заключении диссертации излагают итоги выполненного исследования, рекомендации, перспективы дальнейшей разработки темы.
%% 9.2.3 В заключении автореферата диссертации излагают итоги данного исследования, рекомендации и перспективы дальнейшей разработки темы.
%% Поэтому имеет смысл сделать эту часть общей и загрузить из одного файла в автореферат и в диссертацию:


Результатом этой работы стала система по система по анализу активности собаки, которое способно отслеживать передвижения и позу собаки на стороннем устройстве (как Raspberry или телефон) и предоставлять пользователю ценную статистику о том, как долго его собака спит и ходит в течении дня, без необходимости иметь датчики на самом животном.

Побочным продуктом этого исследования стал датасет, содержащий 4500 изображений, который можно считать трудным для классификации, хотя и предоставляется способ решить эту задачу с 96\% точностью. Помимо этого, предоставлен способ получать данные подобного рода с участием компьютерного зрения. 

Это исследование планировалось как первая ступень к более сложным задачам по распознаванию жестов и сигналов домашних животных. В дальнейшем планируется решить задачу отслеживания конечностей собак.