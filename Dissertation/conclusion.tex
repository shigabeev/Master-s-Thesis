\chapter*{Заключение}						% Заголовок
\addcontentsline{toc}{chapter}{Заключение}	% Добавляем его в оглавление

%% Согласно ГОСТ Р 7.0.11-2011:
%% 5.3.3 В заключении диссертации излагают итоги выполненного исследования, рекомендации, перспективы дальнейшей разработки темы.
%% 9.2.3 В заключении автореферата диссертации излагают итоги данного исследования, рекомендации и перспективы дальнейшей разработки темы.
%% Поэтому имеет смысл сделать эту часть общей и загрузить из одного файла в автореферат и в диссертацию:

В результате данной работы был создан датасет для классификации базовых поз собак. Конечно, метод сбора данных в этой статье спорный, но благодаря нему было возможно создание действительно чистых данных, которые нейронная сеть может успешно распознать. Это позволяет двигаться дальше, имея рабочий прототип. Что касается распознавания более тяжёлых поз - с этими проблемами возможно бороться. Например, в OpenPose\cite{openpose} успешно борются с окклюзиями конечностей людей и множественными объектами, а значит, при определённой упорности, получится и перенести эти технологии в распознавание поз собак.