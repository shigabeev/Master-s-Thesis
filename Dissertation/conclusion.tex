\chapter*{Заключение}						% Заголовок
\addcontentsline{toc}{chapter}{Заключение}	% Добавляем его в оглавление

%% Согласно ГОСТ Р 7.0.11-2011:
%% 5.3.3 В заключении диссертации излагают итоги выполненного исследования, рекомендации, перспективы дальнейшей разработки темы.
%% 9.2.3 В заключении автореферата диссертации излагают итоги данного исследования, рекомендации и перспективы дальнейшей разработки темы.
%% Поэтому имеет смысл сделать эту часть общей и загрузить из одного файла в автореферат и в диссертацию:


Несмотря на трудности в сборе данных, и низкое качество классификации нейронной сети, был проведён анализ ошибок и выявлены следующие:
\begin{itemize}
    \item Для разметки данных надо использовать постоянных сотрудников, которым надо уделить время на то, чтобы разобраться с задачей. Низкая цена обучения не компенсирует многократные перекрытия в надежде найти статистическое среднее.
    \item Количество весов в нейронной сети и количество изображений в датасете должны быть линейно связаны. Хороший размер обучающей выборки для ResNet-34 (2.4 миллиона параметров) - строго от 100 тысяч изображений.
    \item Датасет надо балансировать по количеству изображений в каждом классе.
    \item В данных была огромная внутриклассовая разница между изображениями. У собак множество пород и размеров. Так как точность разметки не идеальна, нейронная сеть может запомнить что собака сидит, когда она чёрная, например. 
\end{itemize}{}
