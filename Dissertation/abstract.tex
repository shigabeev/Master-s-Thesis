\chapter*{\textbf{РЕФЕРАТ}}

Выпускная квалификационная работа, содержит \totalpages\ страницы, \totalfigures\  рисунков, \totaltables\  таблицы, 25 источников.

\noindent Ключевые слова: \MakeUppercase{Компьютерное зрение, Разметка данных, Интеллектуальная разметка данных, Искусственный интеллект, Обучение с частичным применением учителя, Домашние животные, Видео, Интернет вещей}

Объект исследования: активность собак, их перемещения и позы.

Предмет исследования: компьютерное зрение и машинное обучение.

Цель выпускной квалификационной работы: спроектировать систему компьютерного зрения и создать опытный образец её применения.

Задачи выпускной квалификационной работы: создать систему по анализу активности домашнего животного, создать набор обучающих данных для определения позы собаки, проанализировать и подобрать правильную архитектуру нейронных сетей для решения задачи определения позы собак, определить методы повышения производительности на низкопроизводительных устройствах.

Методы исследования: сбор данных, разметка данных, испытание различных архитектур нейронных сетей.

Полученные результаты и их новизна: разработана система для анализа активности домашних животных, получена новая методология по сбору данных для решения поставленной задачи. а также создан набор обучающих данных для обучения нейронных сетей определению позы собаки. 

Практическая значимость: результаты работы позволили создать продукт для анализа активности домашних животных под названием Dogo.


\chapter*{ABSTRACT}

Final qualifying work, \totalpages\ pages, \totalfigures\ figures, \totaltables\ tables,
25 sources.

\noindent Keywords: \MakeUppercase{Computer vision, Pose Estimation, Image Classification, Internet of things, Semi-supervised dataset generation, Data Collection, Artificial Intelligence, Pattern Recognition}

Object of study: dog activity, movements and postures.

Subject of research: computer vision and machine learning.

The purpose of the final qualifying work: to design a computer vision system and to create a prototype of its application.

The tasks of the thesis: create a system to analyze pet activity, create a training dataset to determine dog posture, analyze and select the right neural network architecture to solve the problem of determining dog posture, and determine methods to improve performance on low-performance devices.

Research methods: Data collection, data partitioning, testing different neural network architectures.

The obtained results and their novelty:  a system for analyzing the activity of pets was developed, a new methodology for collecting data to solve the problem was obtained, and a set of training data was created to train neural networks to determine the pose of the dog. 

Practical significance: The results of the work allowed to create a product for the analysis of the activity of pets called Dogo.




% \clearpage
% \phantomsection
% \addcontentsline{toc}{chapter}{\listfigurename}
% \listoffigures									% Список изображений


% %%% Список таблиц %%%
% % (ГОСТ Р 7.0.11-2011, 5.3.10)
% \clearpage
% \phantomsection
% \addcontentsline{toc}{chapter}{\listtablename}
% \listoftables									% Список таблиц
% \newpage