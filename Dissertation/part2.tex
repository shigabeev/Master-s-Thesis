\chapter{ПОСТАНОВКА ЗАДАЧИ} \label{chapt2}

\section{Содержательная постановка задачи} \label{sect2_1}
Задачей данной работы является создание системы компьютерного зрения для определения позы собаки по видеопотоку.

Из литературного обзора можно выделить, что довольно широкий спектр задач компьютерного зрения можно решить с помощью свёрточных нейронных сетей. С другой стороны, нехватка данных для для обучения по выбранной задаче сильно ограничивает применение таких инструментов. Если свести задачу определения позы на видео к задаче классификации каждого изображения собаки на одну из трёх наиболее популярных поз, то для этого потребуется наименьшее количество данных, которые являются самым дорогим компонентом решения этой задачи. Сложность сбора данных обусловлена необходимостью непрерывного контроля их качества и больших временных затрат на их создание.

Как только будут собраны данные, можно будет решить задачу классификации позы собаки на видеопотоке. А затем уже сделать приложение с её использованием. Поэтому задачу определения позы собаки на видеопотоке можно сформулировать в следующем виде.

\textit{Дано: } 

\begin{itemize}[wide]
    \item Устройство, имеющее камеру и позволяющее получить с него изображение или видеопоток.
\end{itemize}

\textit{Найти: }
\begin{itemize}[wide]
    \item Математическую модель (нейронную сеть) классификации изображений, которая позволяла бы определять позу собаки на изображении с точностью выше 95\%;
    \item Требования к обучающим данным (техническое задание), которые позволили бы разметить такие данные и такую нейронную сеть обучить;
    \item Данные для разметки и последующего обучения нейронной сети, разделённые на следующие классы: (\emph{Стоит,  Сидит,  Лежит});
    \item Приложение, которое бы использовало нейронную сеть для классификации изображений в видеопотоке для сбора статистики о поведении домашнего животного, которое работало бы на заданном устройстве с частотой обработки видео выше 1Гц.
\end{itemize}


\section{Постановка задачи классификации изображений}\label{sect2_2}

В данной работе решается задача классификации изображений со следующей формулировкой. 

Имеется видеопоток, который подаётся в систему с частотой $\nu$, измеряемой в Герцах. Необходимо произвести классификацию изображений из видеопотока по заранее известному $N$ количеству классов и вернуть сообщение с номером класса или ответить что изображение не подходит под условия задачи $C$, которые указаны в п.\ref{sect2_1}. 

Допустимо пропускать некоторые изображения, если частота видеопотока превышает частоту работы системы, но частота ответов должна превышать $\nu_{lim}$. 

Изображения, которые не подходят под условия задачи $C$, классифицировать не требуется, но требуется вернуть ответ о том, что они не подходят. 

Допустимо классифицировать изображения с ошибкой, но точность классификации должна быть не меньше $Precision_{lim}$ для конечного набора классифицированных изображений $X$.

В задачи классификации изображений, объекты — это фотографии. Формальная постановка задачи классификации изображений:

$ V = \begin{pmatrix} 
             I_1, & I_2, & ..., & I_n 
      \end{pmatrix}
$
~— множество изображений. 

$ Y = \begin{pmatrix}
                  y_1, & y_2, & ..., & y_n 
      \end{pmatrix}
$
~— конечное множество меток классов.

$y^{*} : V \rightarrow Y$ — неизвестная целевая зависимость, значения которой известны только на объектах конечной обучающей выборки 
\begin{equation} D_m = 
        \begin{pmatrix}
             (I_1, y_1), & ..., & (I_m, y_m)
        \end{pmatrix}
\end{equation}.

Требуется построить алгоритм $a : V \rightarrow Y$ , способный классифицировать объект $I \in V$, который отвечает условиям $C$.


\subsection{Структура входных данных}
Исходными данными в задаче является конечная последовательность кадров, полученная с видеоряда  камеры телефона.

\begin{equation}
    V = \begin{pmatrix}
                I_1, & I_2, & ..., & I_3
        \end{pmatrix}
\end{equation}          

Каждый кадр, или изображение из этого видеоряда является трёхмерной матрицей с количеством столбцов  $h$, количеством строк $w$ и глубиной 3 (красный, зелёный и синий цветовые каналы). 


\subsection{Структура выходных данных}\label{input_struct}
В работе требуется построить оператор $F$  который бы заполнял описанную структуру исходя из поставленной задачи:

\begin{equation}
    F:V \rightarrow RS
\end{equation}

Результаты работы алгоритма заносятся в объект «Состояние собаки»:

\begin{equation}
    RS = \begin{pmatrix}
            Prob, & Loc, & Breed, & Pose 
        \end{pmatrix}
\end{equation}

В этом объекте

$Prob$ - число, вероятность того что на кадре присутствует собака.

$Loc$ - вектор, содержащий информацию об ограничивающей рамке собаки. Состоит из следующих значений:
\begin{equation}
Loc =   \begin{pmatrix}
                x_{min}, & x_{max}, & y_{min}, & y_{max} 
        \end{pmatrix}
\end{equation}

$Breed$ - Вектор для вероятностей принадлежности изображения, внутри $loc$ к каждому из 120 классов пород собаки.

$Pose$ - Вектор вероятности принадлежности изображения внутри $loc$ к каждой из 3 классов поз собаки.

\section{Требования для разметки данных}

В ходе многократных итераций сбора данных и верификации их нейронной сетью, были сформулированны следующие требования, которые позволяют получить наиболее разумный компромисс между широтой применения системы и наивысшим качеством.

\begin{itemize}[wide]
    \item На изображении должна присутствовать собака;
    \item Размер изображения больше 300 пикселей по короткой стороне;
    \item Изображение цветное, трёхканальное;
    \item Фокус изображения должен быть на собаке в кадре.
\end{itemize}

Собака в кадре должна быть:
\begin{itemize}[wide]
    \item Видна целиком;
    \item Ничем не загорожена, даже частично;
    \item На кадре находиться одна;
    \item 4 лапы и голова собаки должны быть видны целиком и не быть загорожены;
    \item Занимать более 20\% кадра по высоте и по ширине;
    \item Собака должна находится в одной из поз, среди массива классов $Y$.
 \end{itemize}
 
Камера относительно собаки:
\begin{itemize}[wide]
    \item Находится на расстоянии более 2м;
    \item Смотрит на собаку на уровне глаз либо выше, но не сверху.
\end{itemize}


\section*{Выводы по разделу 2}
\addcontentsline{toc}{section}{Выводы по разделу 2}
Во второй главе была поставлена задача самой работы, составлено техническое задание на разметку данных, а также поставлена задача классификации изображений. На данном этапе мы имеем весь необходимый базис, чтобы приступить к разметке данных и практической реализации.