%%% Переопределение именований %%%
%\renewcommand{\alsoname}{см. также}
%\renewcommand{\seename}{см.}
%\renewcommand{\headtoname}{вх.}
%\renewcommand{\ccname}{исх.}
%\renewcommand{\enclname}{вкл.}
%\renewcommand{\pagename}{Стр.}
\renewcommand{\partname}{Часть}
\renewcommand{\abstractname}{Аннотация}
\renewcommand{\contentsname}{ОГЛАВЛЕНИЕ} % (ГОСТ Р 7.0.11-2011, 4)
\renewcommand{\figurename}{Рисунок} % (ГОСТ Р 7.0.11-2011, 5.3.9)
\renewcommand{\tablename}{Таблица} % (ГОСТ Р 7.0.11-2011, 5.3.10)
\renewcommand{\indexname}{Предметный указатель}
\renewcommand{\listfigurename}{Список рисунков}
\renewcommand{\listtablename}{Список таблиц}
%\renewcommand{\refname}{\fullbibtitle}
\renewcommand{\bibname}{\fullbibtitle}
%%% Переопределение именований, чтобы можно было и в преамбуле использовать %%%
\renewcommand{\chaptername}{}
\renewcommand{\appendixname}{Приложение} % (ГОСТ Р 7.0.11-2011, 5.7)
% Новые переменные, которые могут использоваться во всём проекте
\newcommand{\authorbibtitle}{Публикации автора по теме диссертации}
\newcommand{\fullbibtitle}{СПИСОК ИСПОЛЬЗОВАННЫХ ИСТОЧНИКОВ} % (ГОСТ Р 7.0.11-2011, 4)
\renewcommand{\bibname}{СПИСОК ИСПОЛЬЗОВАННЫХ ИСТОЧНИКОВ}